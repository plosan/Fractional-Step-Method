
\section{First approach to the FSM}

\subsection{Time integration of the Navier--Stokes equations}

Recall that the Navier--Stokes equations for incompressible and constant viscosity flows are
\begin{equation} \label{eq:navier_stokes_01}
    \left\{
    \begin{aligned}
        \div{\vb{v}} &= 0 \\
        \rho \pdv{\vb{v}}{t} + (\rho \vb{v} \vdot \grad) \vb{v} &= -\grad{p} + \mu \laplacian{\vb{v}}
    \end{aligned}
    \right.
\end{equation}
where $\vb{u} = u \vb{i} + v \vb{j} + w \vb{k}$. By defining the operator
\begin{equation}
    \vb{R}(\vb{v}) = \mu \laplacian{\vb{v}} - (\rho \vb{v} \vdot \grad) \vb{v}
\end{equation}
\eqref{eq:navier_stokes_01} may be rewritten as follows:
\begin{equation} \label{eq:navier_stokes_02}
    \left\{
    \begin{aligned}
        \div{\vb{v}} &= 0 \\
        \rho \pdv{\vb{v}}{t} &= \vb{R}(\vb{v}) - \grad{p}
    \end{aligned}
    \right.
\end{equation}

\colorbox{red}{why integrate the equations?}

Let $[t^n, t^{n+1}] \subset [t_0, t_f]$ be a non--degenerate time interval with $\Delta t = t^{n+1} - t^n$. An implicit integration scheme ($\beta = 1$) is used to integrate the continuity equation with respect to time:
\begin{gather*}
    \int_{t^n}^{t^{n+1}} \div{\vb{v}} \dd{t} =
    \left( \beta \div{\vb{v}}^{n+1} + (1 - \beta) \div{\vb{v}}^{n} \right) \, \Delta t =
    \div{\vb{v}}^{n+1} \, \Delta t = 0
\end{gather*}
Since $\Delta t > 0$, it follows that
\begin{equation} \label{eq:continuity_time_integrated}
    \div{\vb{v}}^{n+1} = 0
\end{equation}
which is the time--integrated continuity equation. As for the momentum equation,
\begin{equation*}
    \int_{t^n}^{t^{n+1}} \rho \pdv{\vb{v}}{t} \dd{t} =
    \int_{t^n}^{t^{n+1}} \vb{R}(\vb{v}) \dd{t} -
    \int_{t^n}^{t^{n+1}} \grad{p} \dd{t}
\end{equation*}
The left--hand side computation is straightforward as the density is constant and the fundamental theorem of calculus is applied:
\begin{equation*}
    \int_{t^n}^{t^{n+1}} \rho \pdv{\vb{v}}{t} \dd{t} =
    \rho \int_{t^n}^{t^{n+1}} \pdv{\vb{v}}{t} \dd{t} =
    \rho ( \vb{v}^{n+1} - \vb{v}^n )
\end{equation*}
In order to integrate $\vb{R}(\vb{v})$, define
\begin{equation*}
    \mathfrak{R}(\vb{v},t) = \int_{t_0}^s \vb{R}(\vb{v}) \dd{s}
\end{equation*}
so that
\begin{equation} \label{eq:integral_Rv}
    \mathfrak{R}(\vb{v},t^{n+1}) - \mathfrak{R}(\vb{v},t^{n}) =
    \int_{t^n}^{t^{n+1}} \vb{R}(\vb{v}) \dd{t}
\end{equation}
By applying the two--step Adams--Bashforth method, \eqref{eq:integral_Rv} results in
\begin{equation*}
    \int_{t^n}^{t^{n+1}} \vb{R}(\vb{v}) \dd{t} =
    \left( \frac{3}{2} \vb{R}(\vb{v}^{n+1}) - \frac{1}{2} \vb{R}(\vb{v}^n) \right) \, \Delta t
\end{equation*}
Again the implicit integration scheme ($\beta = 1$) is used to compute the third term:
\begin{equation*}
    \int_{t^n}^{t^{n+}} \grad{p} \dd{t} =
    \left( \beta \grad{p}^{n+1} + (1 - \beta) \grad{p}^n \right) \, \Delta t =
    \grad{p}^{n+1} \, \Delta t
\end{equation*}
Rearranging terms yields the time--integrated momentum equation:
\begin{equation}
    \rho \frac{\vb{v}^{n+1} - \vb{v}^n}{\Delta t} =
    \frac{3}{2} \vb{R}(\vb{v}^{n+1}) - \frac{1}{2} \vb{R}(\vb{v}^n) - \grad{p}^{n+1}
\end{equation}
