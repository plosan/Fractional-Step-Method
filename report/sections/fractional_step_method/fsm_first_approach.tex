
\section{First approach to the FSM}

\subsection{Time integration of the Navier--Stokes equations}

<<<<<<< HEAD
Recall that the Navier--Stokes equations for incompressible and constant viscosity flows are
=======
Recall that the Navier--Stokes equations for incompressible and constant viscosity flows are given by
>>>>>>> Changes in report. Changes in code: redefinition of surfX, surfY, vol
\begin{equation} \label{eq:navier_stokes_01}
    \left\{
    \begin{aligned}
        \div{\vb{v}} &= 0 \\
        \rho \pdv{\vb{v}}{t} + (\rho \vb{v} \vdot \grad) \vb{v} &= -\grad{p} + \mu \laplacian{\vb{v}}
    \end{aligned}
    \right.
\end{equation}
<<<<<<< HEAD
where $\vb{u} = u \vb{i} + v \vb{j} + w \vb{k}$. By defining the operator
\begin{equation}
    \vb{R}(\vb{v}) = \mu \laplacian{\vb{v}} - (\rho \vb{v} \vdot \grad) \vb{v}
\end{equation}
\eqref{eq:navier_stokes_01} may be rewritten as follows:
=======
where $\rho$ and $\mu$ are the fluid density and viscosity, respectively, $p \colon \Omega \times I \rightarrow \real$ is the pressure field and $\vb{v} = (u, v, w) \colon \Omega \times I \rightarrow \real^3$ is the velocity field. The operator
\begin{equation}
    \vb{R}(t,\vb{v}) = 
    \mu \laplacian{\vb{v}(\cdot,t)} - (\rho \vb{v}(\cdot,t) \vdot \grad) \vb{v}(\cdot,t) =
    \mu \laplacian{\vb{v}} - (\rho \vb{v} \vdot \grad) \vb{v}
\end{equation}
permits rewriting Equation \eqref{eq:navier_stokes_01} as follows:
>>>>>>> Changes in report. Changes in code: redefinition of surfX, surfY, vol
\begin{equation} \label{eq:navier_stokes_02}
    \left\{
    \begin{aligned}
        \div{\vb{v}} &= 0 \\
<<<<<<< HEAD
        \rho \pdv{\vb{v}}{t} &= \vb{R}(\vb{v}) - \grad{p}
    \end{aligned}
    \right.
\end{equation}

\colorbox{red}{why integrate the equations?}

Let $[t^n, t^{n+1}] \subset [t_0, t_f]$ be a non--degenerate time interval with $\Delta t = t^{n+1} - t^n$. An implicit integration scheme ($\beta = 1$) is used to integrate the continuity equation with respect to time:
\begin{gather*}
    \int_{t^n}^{t^{n+1}} \div{\vb{v}} \dd{t} =
    \left( \beta \div{\vb{v}}^{n+1} + (1 - \beta) \div{\vb{v}}^{n} \right) \, \Delta t =
    \div{\vb{v}}^{n+1} \, \Delta t = 0
\end{gather*}
Since $\Delta t > 0$, it follows that
\begin{equation} \label{eq:continuity_time_integrated}
    \div{\vb{v}}^{n+1} = 0
=======
        \rho \pdv{\vb{v}}{t} &= \vb{R}(t,\vb{v}) - \grad{p}
    \end{aligned}
    \right.
\end{equation}
Therefore the momentum equation is now an evolution equation.

In order to solve \eqref{eq:navier_stokes_02} numerically in $\Omega \times I \subset \real^4$ with suitable boundary conditions on $\partial \Omega$ and initial conditions on $\Omega \times \{ t = 0 \}$ following a finite volume method, it is necessary to discretise $\Omega$ in control volumes and integrate \eqref{eq:navier_stokes_02} over time intervals of the form $[t^n, t^{n+1}]$. 


Let $[t^n, t^{n+1}] \subset [t_0, t_f] = \overline{I}$ be a non--degenerate time interval and $\Delta t = t^{n+1} - t^n$. An implicit integration scheme ($\beta = 1$) is used to integrate the continuity equation with respect to time:
\begin{gather*}
    \int_{t^n}^{t^{n+1}} \div{\vb{v}} \dd{t} =
    \left( \beta (\div{\vb{v}})^{n+1} + (1 - \beta) (\div{\vb{v}})^{n} \right) \, \Delta t =
    (\div{\vb{v}})^{n+1} \, \Delta t = 0
\end{gather*}
Since $\Delta t > 0$, it follows that
\begin{equation} \label{eq:continuity_time_integrated}
    (\div{\vb{v}})^{n+1} \approx \div{\vb{v}^{n+1}} = 0
>>>>>>> Changes in report. Changes in code: redefinition of surfX, surfY, vol
\end{equation}
which is the time--integrated continuity equation. As for the momentum equation,
\begin{equation*}
    \int_{t^n}^{t^{n+1}} \rho \pdv{\vb{v}}{t} \dd{t} =
    \int_{t^n}^{t^{n+1}} \vb{R}(\vb{v}) \dd{t} -
    \int_{t^n}^{t^{n+1}} \grad{p} \dd{t}
\end{equation*}
<<<<<<< HEAD
The left--hand side computation is straightforward as the density is constant and the fundamental theorem of calculus is applied:
=======
For the left--hand side, it is known that the density is constant, hence:
>>>>>>> Changes in report. Changes in code: redefinition of surfX, surfY, vol
\begin{equation*}
    \int_{t^n}^{t^{n+1}} \rho \pdv{\vb{v}}{t} \dd{t} =
    \rho \int_{t^n}^{t^{n+1}} \pdv{\vb{v}}{t} \dd{t} =
    \rho ( \vb{v}^{n+1} - \vb{v}^n )
\end{equation*}
In order to integrate $\vb{R}(\vb{v})$, define
\begin{equation*}
<<<<<<< HEAD
    \mathfrak{R}(\vb{v},t) = \int_{t_0}^s \vb{R}(\vb{v}) \dd{s}
\end{equation*}
so that
\begin{equation} \label{eq:integral_Rv}
    \mathfrak{R}(\vb{v},t^{n+1}) - \mathfrak{R}(\vb{v},t^{n}) =
    \int_{t^n}^{t^{n+1}} \vb{R}(\vb{v}) \dd{t}
\end{equation}
By applying the two--step Adams--Bashforth method, \eqref{eq:integral_Rv} results in
\begin{equation*}
    \int_{t^n}^{t^{n+1}} \vb{R}(\vb{v}) \dd{t} =
    \left( \frac{3}{2} \vb{R}(\vb{v}^{n+1}) - \frac{1}{2} \vb{R}(\vb{v}^n) \right) \, \Delta t
\end{equation*}
Again the implicit integration scheme ($\beta = 1$) is used to compute the third term:
=======
    \mathfrak{R}(t,\vb{v}) = \int_{0}^t \vb{R}(s,\vb{v}) \dd{s}
\end{equation*}
so that
\begin{equation} \label{eq:integral_Rv}
    \mathfrak{R}(t^{n+1}, \vb{v}) - \mathfrak{R}(t^n, \vb{v}) =
    \int_{t^n}^{t^{n+1}} \vb{R}(s,\vb{v}) \dd{s}
\end{equation}
By applying the two--step Adams--Bashforth method, \eqref{eq:integral_Rv} results in
\begin{equation*}
    \int_{t^n}^{t^{n+1}} \vb{R}(s,\vb{v}) \dd{t} =
    \left( \frac{3}{2} \vb{R}(t^{n}, \vb{v}) - \frac{1}{2} \vb{R}(t^{n-1}, \vb{v}) \right) \, \Delta t
\end{equation*}
For the sake of simplicity in notation, it shall be written $\vb{R}(\vb{v}^n)$ in place of $\vb{R}(t^n, \vb{v})$. Regarding the integral of the pressure gradient, again the implicit integration scheme ($\beta = 1$) is used, which yields
>>>>>>> Changes in report. Changes in code: redefinition of surfX, surfY, vol
\begin{equation*}
    \int_{t^n}^{t^{n+}} \grad{p} \dd{t} =
    \left( \beta \grad{p}^{n+1} + (1 - \beta) \grad{p}^n \right) \, \Delta t =
    \grad{p}^{n+1} \, \Delta t
\end{equation*}
<<<<<<< HEAD
Rearranging terms yields the time--integrated momentum equation:
\begin{equation}
    \rho \frac{\vb{v}^{n+1} - \vb{v}^n}{\Delta t} =
    \frac{3}{2} \vb{R}(\vb{v}^{n+1}) - \frac{1}{2} \vb{R}(\vb{v}^n) - \grad{p}^{n+1}
\end{equation}
=======
By rearranging terms the time--integrated momentum equation is obtained:
\begin{equation} \label{eq:momentum_time_integrated}
    \rho \frac{\vb{v}^{n+1} - \vb{v}^n}{\Delta t} =
    \frac{3}{2} \vb{R}(\vb{v}^{n}) - \frac{1}{2} \vb{R}(\vb{v}^{n-1}) - \grad{p}^{n+1}
\end{equation}

\subsection{Application of the Helmholtz decomposition to the Navier--Stokes equations}

Define the predictor velocity by
\begin{equation} \label{eq:predictor_velocity}
    \vb{v}^p = \vb{v}^{n+1} + \frac{\Delta t}{\rho} \grad{p^{n+1}}
\end{equation}

\colorbox{red}{Since the velocity field is divergence--less, the predictor velocity is decomposed into... which is unique}

After introducing \eqref{eq:predictor_velocity} into \eqref{eq:momentum_time_integrated}, the time--integrated momentum equation becomes the velocity projection equation:
\begin{equation} \label{eq:momentum_time_integrated_predictor_velocity}
    \rho \frac{\vb{v}^p - \vb{v}^n}{\Delta t} = 
    \frac{3}{2} \vb{R}(\vb{v}^n) - \frac{1}{2} \vb{R}(\vb{v}^{n-1})
\end{equation}

Taking the divergence of both sides of \eqref{eq:predictor_velocity} and using \eqref{eq:continuity_time_integrated} results in
\begin{equation}
    \div{\vb{v}^p} = 
    \div{\vb{v}^{n+1}} + \div(\frac{\Delta t}{\rho} \grad{p}^{n+1}) = 
    \frac{\Delta t}{\rho} \laplacian{p^{n+1}}
\end{equation}
which is a Poisson equation for the pressure term:
\begin{equation} \label{eq:poisson_equation_pressure}
    \laplacian{p^{n+1}} = \frac{\rho}{\Delta t} \div{\vb{v}^p}
\end{equation}

\subsection{Solving algorithm}

Notice that from Equation \eqref{eq:momentum_time_integrated_predictor_velocity} the predictor velocity may also be calculated by
\begin{equation} \label{eq:predictor_velocity_2}
    \vb{v}^p = 
    \vb{v}^n + 
    \frac{\Delta t}{\rho} \left( \frac{3}{2} \vb{R}(\vb{v}^n) - \frac{1}{2} \vb{R}(\vb{v}^{n-1}) \right)
\end{equation}
Assume that the data up to time $t^n$ is known, that is to say, $\vb{R}(\vb{v}^{n-1}), \vb{R}(\vb{v}^n)$ and $\vb{v}^n$ are known. Then the computation of the predictor velocity is immediate by means of Equation \eqref{eq:predictor_velocity_2}. By integrating the laplacian of the pressure term at $t^{n+1}$ on each control volume, equation \eqref{eq:poisson_equation_pressure} may be transformed into a linear equation which, with suitable boundary conditions on $\partial \Omega$, allows us to compute $p^{n+1}$ (this matter is carried out in Section \colorbox{red}{section}). As a result, both $\vb{v}^p$ and $\grad{p}^{n+1}$ are known, whereby the velocity field at $t^{n+1}$ is obtained from Equation \eqref{eq:predictor_velocity}:
\begin{equation} \label{eq:velocity_at_next_instant}
    \vb{v}^{n+1} = \vb{v}^p - \frac{\Delta t}{\rho} \grad{p^{n+1}}
\end{equation} 
This procedure is summarised in the following algorithm:
\begin{algorithm}[ht]
	\caption{Computation of the velocity field at $t^{n+1}$.}
	\label{algorithm:computation_of_velocity_field_at_next_instant}
	\begin{algorithmic}[0]
		\State 
		\begin{enumerate}[label=\textbf{\arabic*},topsep=0pt]
			\item Evaluation of $\vb{R}(\vb{v}^n)$.
			\item Evaluation of the predictor velocity $\vb{v}^p$ with Equation \eqref{eq:predictor_velocity_2}.
			\item Evaluation of the pressure at $t^{n+1}$ by solving Equation \eqref{eq:poisson_equation_pressure}.
			\item Evaluation of the velocity at $t^{n+1}$ with Equation \eqref{eq:velocity_at_next_instant}.
		\end{enumerate}
	\end{algorithmic}
\end{algorithm}

Despite of the simplicity Algorithm \ref{algorithm:computation_of_velocity_field_at_next_instant}, the produced velocity and pressure fields entail a fundamental problem, namely, these have no physical meaning. In the section below, an example of this is exposed.




>>>>>>> Changes in report. Changes in code: redefinition of surfX, surfY, vol
