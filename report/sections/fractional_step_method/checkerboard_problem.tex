<<<<<<< HEAD
\subsection{The checkerboard problem}

Notice that \eqref{eq:velocity_time_n+1_fsm_first_approach} is a vector equation, hence expanding it by components gives:
\begin{equation} \label{eq:velocity_checkerboard_problem}
    \left\{
        \begin{aligned}
            u^{n+1} &= u^p - \frac{\Delta t}{\rho} p_x^{n+1} \\
            v^{n+1} &= v^p - \frac{\Delta t}{\rho} p_y^{n+1} \\
            w^{n+1} &= w^p - \frac{\Delta t}{\rho} p_z^{n+1} \\
        \end{aligned}
    \right.
\end{equation}
Consider the discretization of the $x$--component of \eqref{eq:velocity_checkerboard_problem}:
\begin{equation}
    u_P^{n+1} = 
    u_P^p - \frac{\Delta t}{\rho} 
    \left( \frac{p_E^{n+1} - p_W^{n+1}}{2 \, \Delta x} \right)
\end{equation}
Notice that the discrete approximation of $\grad{p}^{n+1}$ is independent of $p_P^{n+1}$.

=======
\section{The checkerboard problem}

Consider the $x$--component of Equation \eqref{eq:velocity_at_next_instant}
\begin{equation} \label{eq:velocity_at_next_instant_x_component}
    u^{n+1} = 
    u^p - \frac{\Delta t}{\rho} \left( \pdv{p}{x} \right)^{n+1} \approx
    u^p - \frac{\Delta t}{\rho} \pdv{p^{n+1}}{x}
\end{equation}
and the following discretisation in the $x$--direction of a neighbourhood of a point $P$:
>>>>>>> Changes in report. Changes in code: redefinition of surfX, surfY, vol
\begin{figure}[h]
    \centering
    \begin{tikzpicture}
        \def\side{2cm}
        \def\ys{0.5cm}
        \def\ofset{0.1cm}
<<<<<<< HEAD
        \def\arrowlength{0.75cm}
=======
        \def\velocitylength{0.75cm}
>>>>>>> Changes in report. Changes in code: redefinition of surfX, surfY, vol
        % Control volume
        \fill[cvfill] (0,0) rectangle (3*\side,\side);
        \draw[wall] (0,0) rectangle (3*\side,\side);
        \draw[wall] (\side,0) -- ++(0,\side);
        \draw[wall] (2*\side,0) -- ++(0,\side);
        % Node W
<<<<<<< HEAD
        \draw[velocity] (0.5*\side,0.5*\side) -- node[midway, below]{$u_W^{n+1}$} ++(\arrowlength,0);
        \filldraw[black] (0.5*\side,0.5*\side) circle (2pt);
        \node[black, yshift=+\ys] at (0.5*\side,0.5*\side) {$p_W^{n+1}$};
        % Node P
        \draw[velocity] (1.5*\side,0.5*\side) -- node[midway, below]{$u_P^{n+1}$} ++(\arrowlength,0);
        \filldraw[black] (1.5*\side,0.5*\side) circle (2pt);
        \node[black, yshift=+\ys] at (1.5*\side,0.5*\side) {$p_P^{n+1}$};
        % Node E
        \draw[velocity] (2.5*\side,0.5*\side) -- node[midway, below]{$u_E^{n+1}$} ++(\arrowlength,0);
=======
        \draw[velocity] (0.5*\side,0.5*\side) -- node[midway, below]{$u_W^{n+1}$} ++(\velocitylength,0);
        \filldraw[black] (0.5*\side,0.5*\side) circle (2pt);
        \node[black, yshift=+\ys] at (0.5*\side,0.5*\side) {$p_W^{n+1}$};
        % Node P
        \draw[velocity] (1.5*\side,0.5*\side) -- node[midway, below]{$u_P^{n+1}$} ++(\velocitylength,0);
        \filldraw[black] (1.5*\side,0.5*\side) circle (2pt);
        \node[black, yshift=+\ys] at (1.5*\side,0.5*\side) {$p_P^{n+1}$};
        % Node E
        \draw[velocity] (2.5*\side,0.5*\side) -- node[midway, below]{$u_E^{n+1}$} ++(\velocitylength,0);
>>>>>>> Changes in report. Changes in code: redefinition of surfX, surfY, vol
        \filldraw[black] (2.5*\side,0.5*\side) circle (2pt);
        \node[black, yshift=+\ys] at (2.5*\side,0.5*\side) {$p_E^{n+1}$};
        % Measuring line
        \draw[measureline] (0.5*\side,-0.5*\side) -- node[above]{$\Delta x$} ++(\side,0);
        \draw[measureline] (1.5*\side,-0.5*\side) -- node[above]{$\Delta x$} ++(\side,0);
        \draw[measureaux, yshift=-\ofset] (0.5*\side,-0.5*\side) -- ++(0,0.5*\side);
        \draw[measureaux, yshift=-\ofset] (1.5*\side,-0.5*\side) -- ++(0,0.5*\side);
        \draw[measureaux, yshift=-\ofset] (2.5*\side,-0.5*\side) -- ++(0,0.5*\side);
    \end{tikzpicture}
<<<<<<< HEAD
\end{figure}

=======
    \caption{}
    \label{fig:checkerboard_problem_figure_1}
\end{figure}

By applying finite differences at $P$ to compute the partial derivative in Equation \eqref{eq:velocity_at_next_instant_x_component}, one may approximate $u_P^{n+1}$ as follows:
\begin{equation}
    u_P^{n+1} = u^p - \frac{\Delta t}{\rho} \frac{p_E^{n+1} - p_W^{n+1}}{2 \Delta x}
\end{equation}
We notice that $u_P^{n+1}$ does not depend upon $p_P^{n+1}$. The same is true for the $y$ and $z$--components of \eqref{eq:velocity_at_next_instant}. This is a consequence of the pressure $p_P^{n+1}$ and the gradient $\grad{p}^{n+1}$ being decoupled, \ie independent of one another.

This leads to converged velocity fields for unphysical pressure distributions. For instance, consider the following situation
>>>>>>> Changes in report. Changes in code: redefinition of surfX, surfY, vol

\begin{figure}[h]
    \centering
    \begin{tikzpicture}
        \def\side{2cm}
        \def\ys{0.5cm}
        \def\ofset{0.1cm}
<<<<<<< HEAD
        \def\arrowlength{0.75cm}
=======
        \def\velocitylength{0.75cm}
>>>>>>> Changes in report. Changes in code: redefinition of surfX, surfY, vol
        % Control volumes
        \fill[cvfill] (0,0) rectangle (5*\side,\side);
        \draw[wall] (0,0) rectangle (5*\side,\side);
        \draw[wall] (\side,0) -- ++(0,\side);
        \draw[wall] (2*\side,0) -- ++(0,\side);
        \draw[wall] (3*\side,0) -- ++(0,\side);
        \draw[wall] (4*\side,0) -- ++(0,\side);



        % Node WW
        \begin{scope}[shift={(0.5*\side, 0.5*\side)}]
<<<<<<< HEAD
            \draw[velocity] (0,0) -- node[midway, below]{$u_{WW}^{n+1}$} ++(\arrowlength,0);
            \filldraw[black] (0,0) circle (2pt);
            \node[black, yshift=+\ys] at (0,0) {$p_{WW}^{n+1}$};
        \end{scope}

        
        % Node P
        \draw[velocity] (1.5*\side,0.5*\side) -- node[midway, below]{$u_P^{n+1}$} ++(\arrowlength,0);
        \filldraw[black] (1.5*\side,0.5*\side) circle (2pt);
        \node[black, yshift=+\ys] at (1.5*\side,0.5*\side) {$p_P^{n+1}$};
        % Node E
        \draw[velocity] (2.5*\side,0.5*\side) -- node[midway, below]{$u_E^{n+1}$} ++(\arrowlength,0);
        \filldraw[black] (2.5*\side,0.5*\side) circle (2pt);
        \node[black, yshift=+\ys] at (2.5*\side,0.5*\side) {$p_E^{n+1}$};
    \end{tikzpicture}
\end{figure}
=======
            \draw[velocity] (0,0) -- node[midway, below]{$u_{WW}^{n+1}$} ++(\velocitylength,0);
            \filldraw[black] (0,0) circle (2pt);
            \node[black, yshift=+\ys] at (0,0) {$p_{WW}^{n+1}$};
        \end{scope}
        
        % Node W
        \begin{scope}[shift={(1.5*\side,0.5*\side)}]
            \draw[velocity] (0,0) -- node[midway, below]{$u_{W}^{n+1}$} ++(\velocitylength,0);
            \filldraw[black] (0,0) circle (2pt);
            \node[black, yshift=+\ys] at (0,0) {$p_{W}^{n+1}$};
        \end{scope}
        
        % Node P
        \begin{scope}[shift={(2.5*\side,0.5*\side)}]
            \draw[velocity] (0,0) -- node[midway, below]{$u_{P}^{n+1}$} ++(\velocitylength,0);
            \filldraw[black] (0,0) circle (2pt);
            \node[black, yshift=+\ys] at (0,0) {$p_{P}^{n+1}$};
        \end{scope}
        
        % Node E
        \begin{scope}[shift={(3.5*\side,0.5*\side)}]
            \draw[velocity] (0,0) -- node[midway, below]{$u_{E}^{n+1}$} ++(\velocitylength,0);
            \filldraw[black] (0,0) circle (2pt);
            \node[black, yshift=+\ys] at (0,0) {$p_{E}^{n+1}$};
        \end{scope}
        
        % Node EE
        \begin{scope}[shift={(4.5*\side,0.5*\side)}]
            \draw[velocity] (0,0) -- node[midway, below]{$u_{EE}^{n+1}$} ++(\velocitylength,0);
            \filldraw[black] (0,0) circle (2pt);
            \node[black, yshift=+\ys] at (0,0) {$p_{EE}^{n+1}$};            
        \end{scope}

        \begin{scope}[shift={(5*\side,0.5*\side)}]
            \node[anchor=west] at (0,0) {
                \begin{tabular}{c}
                    $p_{WW}^{n+1} = p_{P}^{n+1} = p_{EE}^{n+1} = 100$ \\
                    $p_{W}^{n+1} = p_{E}^{n+1} = 0$
                \end{tabular}
            };
        \end{scope}
    \end{tikzpicture}
    \caption{}
    \label{fig:checkerboard_problem_figure_2}
\end{figure}

which is a checkerboard pattern as far as the pressure distribution is concerned. The partial derivative of pressure satisfies $\partial_x p^{n+1} = 0$ at nodes $W$, $P$ and $E$, despite being a pressure distribution with no physical sense, thus yielding incorrect converged velocities at these nodes. It is clear that a more sophisticated strategy to relate the velocity and pressure fields is needed. 



>>>>>>> Changes in report. Changes in code: redefinition of surfX, surfY, vol
